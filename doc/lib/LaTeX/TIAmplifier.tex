\documentclass[a4paper]{article}
\usepackage[top=1in, bottom=1.25in, left=1.25in, right=1.25in]{geometry}
\usepackage{amsmath}
\usepackage{multicol}
\usepackage{graphicx}
\usepackage[utf8]{inputenc}
\usepackage[english]{babel}
\setlength{\parskip}{0.03cm plus4mm minus3mm}
\RequirePackage{ltxcmds}[2010/12/07]

\usepackage{hyperref}
%opening
\title{TI Amplifier}

\begin{document}

\maketitle

This block has one input signal and one output signal both corresponding to electrical signals. The output signal corresponds to the amplification of the input signal with added noise.


\subsection*{Input Parameters}

\begin{itemize}
	\item amplification\{1e6\}
	\item noiseamp\{ 1e-4 \}
\end{itemize}

\subsection*{Methods}
 
TIAmplifier() {}
\bigbreak
TIAmplifier(vector$<$Signal *$>$ \&InputSig, vector$<$Signal *$>$ \&OutputSig) :Block(InputSig, OutputSig) {}
\bigbreak
void initialize(void)
\bigbreak
bool runBlock(void)
\bigbreak
void setAmpplification(\texttt{t\_real} Amplification)
\bigbreak
void setNoiseAmpplitude(\texttt{t\_real} NoiseAmplitude)

\subsection*{Functional description}

The output signal is the product of the input signal with the parameter \textit{amplification} plus a component that corresponds to the noise introduced by the amplification of the signal. 

\pagebreak

\subsection*{Input Signals}

\subparagraph*{Number:} 1

\subparagraph*{Type:} Electrical (TimeContinuousAmplitudeContinuousReal)

\subsection*{Output Signals}

\subparagraph*{Number:} 1

\subparagraph*{Type:} Electrical (TimeContinuousAmplitudeContinuousReal)

\subsection*{Examples} 

\subsection*{Sugestions for future improvement}


\end{document}