\documentclass[a4paper]{article}
\usepackage[top=1in, bottom=1.25in, left=1.25in, right=1.25in]{geometry}
\usepackage{amsmath}
\usepackage{multicol}
\usepackage{graphicx}
\usepackage[utf8]{inputenc}
\usepackage[english]{babel}
\setlength{\parskip}{0.03cm plus4mm minus3mm}
\RequirePackage{ltxcmds}[2010/12/07]

\usepackage{hyperref}
%opening
\title{Binary Source}

\begin{document}

\maketitle
This block generates a sequence of binary values (1 or 0) and it can work in four different modes: 

\begin{multicols}{2}
\begin{enumerate}
	\item Random
	\item PseudoRandom 
	\item DeterministicCyclic 
	\item DeterministicAppendZeros 
\end{enumerate}
\end{multicols}

This blocks doesn't accept any input signal. It produces any number of output signals.

\subsection*{Input Parameters}

	\begin{itemize}
		\item mode\{PseudoRandom\}\linebreak
		(Random, PseudoRandom, DeterministicCyclic, DeterministicAppendZeros)
		\item probabilityOfZero\{0.5\}\linebreak
		(real $\in$ [0,1])
		\item patternLength\{7\} \linebreak
		(integer $\in$ [1,32]) 
		\item bitStream\{"0100011101010101"\} \linebreak
		(string of 0's and 1's)
		\item numberOfBits\{-1\} \linebreak
		(long int)
		\item bitPeriod\{1.0/100e9\} \linebreak
		(double)
	\end{itemize}

\subsection*{Methods}

BinarySource(vector$\langle$Signal *$\rangle$ \&InputSig, vector$\langle$Signal *$\rangle$ \&OutputSig) :Block(InputSig, OutputSig)\{\};
\bigbreak	 
void initialize(void);
\bigbreak	 
bool runBlock(void);
\bigbreak	 
void setMode(BinarySourceMode m) {mode = m;}
BinarySourceMode const getMode(void) \{ return mode; \};
\bigbreak	 
void setProbabilityOfZero(double pZero) \{ probabilityOfZero = pZero; \};
\bigbreak
double const getProbabilityOfZero(void) \{ return probabilityOfZero; \};
\bigbreak	 
void setBitStream(string bStream) \{ bitStream = bStream; \};
\bigbreak
string const getBitStream(void) \{ return bitStream; \};
\bigbreak	 
void setNumberOfBits(long int nOfBits) \{ numberOfBits = nOfBits; \};
\bigbreak
long int const getNumberOfBits(void) \{ return numberOfBits; \};
\bigbreak	 
void setPatternLength(int pLength) \{ patternLength = pLength; \};
\bigbreak
int const getPatternLength(void) \{ return patternLength; \}
\bigbreak	 
void setBitPeriod(double bPeriod);
\bigbreak
double const getBitPeriod(void) \{ return bitPeriod; \}

\subsection*{Functional description}

The \textit{mode} parameter allows the user to select between one of the four operation modes of the binary source.

\subparagraph*{Random Mode}
Generates a 0 with probability \textit{probabilityOfZero} and a 1 with probability 1-\textit{probabilityOfZero}.

\subparagraph*{Pseudorandom Mode} 
Generates a pseudorandom sequence with period $2^\textit{patternLength}-1$.

\subparagraph*{DeterministicCyclic Mode}
Generates the sequence of 0's and 1's specified by \textit{bitStream} and then repeats it.

\subparagraph*{DeterministicAppendZeros Mode}
Generates the sequence of 0's and 1's specified by \textit{bitStream} and then it fills the rest of the buffer space with zeros.

\subsection*{Input Signals}


\subparagraph*{Number:} 0

\subparagraph*{Type:} Binary (DiscreteTimeDiscreteAmplitude)

\subsection*{Output Signals}

\subparagraph*{Number:} 1 or more

\subparagraph*{Type:} Binary (DiscreteTimeDiscreteAmplitude)

\subsection*{Examples} 

\paragraph*{Random Mode}

\paragraph*{PseudoRandom Mode}
As an example consider a pseudorandom sequence with \textit{patternLength}=3 which contains a total of 7 ($2^3-1$) bits. In this sequence it is possible to find every combination of 0's and 1's that compose a 3 bit long subsequence with the exception of $000$. For this example the possible subsequences are $010$, $110$, $101$, $100$, $111$, $001$ and $100$ (they appear in figure \ref{BinarySequenceN3} numbered in this order). Some of these require wrap. 

\begin{figure}[h]
	\centering
\includegraphics[width=0.5\textwidth]{BinarySequenceN3}

\caption{Example of a pseudorandom sequence with a pattern length equal to 3.}\label{BinarySequenceN3}
\end{figure}

\paragraph*{DeterministicCyclic Mode}

As an example take the \textit{bit stream} '0100011101010101'. The generated binary signal is displayed in.

\paragraph*{DeterministicAppendZeros Mode}

Take as an example the \textit{bit stream} '0100011101010101'. The generated binary signal is displayed in \ref{MQAM1_DeterministAppendZeros}.

\begin{figure}
	\centering
	\includegraphics[width=\textwidth]{MQAM1_DeterministAppendZeros}
	
	\caption{Binary signal generated by the block operating in the \textit{PseudoRandom} mode}\label{MQAM1_DeterministAppendZeros}
\end{figure}

\subsection*{Sugestions for future improvement}

Implement an input signal that can work as trigger.

\end{document}