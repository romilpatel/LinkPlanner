\documentclass[a4paper]{article}
\usepackage[top=1in, bottom=1.25in, left=1.25in, right=1.25in]{geometry}
\usepackage{amsmath}
\usepackage{multicol}
\usepackage{graphicx}
\RequirePackage{ltxcmds}[2010/12/07]
%opening
\title{M-QAM Mapper}

\begin{document}
	
\maketitle

This block does the mapping of the binary signal using a \textit{m}-QAM modulation. It accepts one input signal of the binary type and it produces two output signals which are a sequence of 1's and -1's.

\subsection*{Input Parameters}

\begin{itemize}
	\item m\{4\} \linebreak
	(m should be of the form $2^n$ with n integer)
	\item iqAmplitudes\{\{ 1.0, 1.0 \}, \{ -1.0, 1.0 \}, \{ -1.0, -1.0 \}, \{ 1.0, -1.0 \}\} \linebreak
	
\end{itemize}

\subsection*{Methods}

MQamMapper(vector$<$Signal *$>$ \&InputSig, vector$<$Signal *$>$ \&OutputSig) :Block(InputSig, OutputSig) \{\};
\bigbreak	
void initialize(void);
\bigbreak	
bool runBlock(void);
\bigbreak	
void setM(int mValue);
\bigbreak	
void setIqAmplitudes(vector$<$t\_iqValues$>$ iqAmplitudesValues);

\subsection*{Functional Description}

In the case of m=4 this block atributes to each pair of bits a point in the I-Q space. The constellation used is defined by the \textit{iqAmplitudes} vector. 

\subsection*{Input Signals}

\textbf{Number}: 1

\textbf{Type}: Binary (DiscreteTimeDiscreteAmplitude)

\subsection*{Output Signals}

\textbf{Number}: 2

\textbf{Type}: Sequence of 1's and -1's (DiscreteTimeDiscreteAmplitude)

\subsection*{Example}

%\begin{figure}[h]
%	\includegraphics[width=\textwidth]{MQAM2}
%\end{figure}

\subsection*{Sugestions for future improvement}

\end{document}