\documentclass[a4paper]{article}
\usepackage[top=1in, bottom=1.25in, left=1.25in, right=1.25in]{geometry}
\usepackage{amsmath}
\usepackage{multicol}
\usepackage{graphicx}
\usepackage[utf8]{inputenc}
\usepackage[english]{babel}
\setlength{\parskip}{0.03cm plus4mm minus3mm}
\RequirePackage{ltxcmds}[2010/12/07]

\usepackage{hyperref}
%opening
\title{Sampler}

\begin{document}

\maketitle

This block accepts one real electrical continuous in time signal and outputs a real electrical discrete in time signal. The output signal is obtained by sampling the input signal with a predetermined samplig rate.

\subsection*{Input Parameters}

\begin{itemize}
	\item samplesToSkip\{ 0 \}
\end{itemize}

\subsection*{Methods}
 
Sampler() {}
\bigbreak
Sampler(vector$<$Signal *$>$ \&InputSig, vector$<$Signal *$>$ \&OutputSig) :Block(InputSig, OutputSig) {}
\bigbreak
void initialize(void)
\bigbreak
bool runBlock(void)
\bigbreak
void setSamplesToSkip(\texttt{t\_integer} sToSkip)

\subsection*{Functional description}


\pagebreak

\subsection*{Input Signals}

\subparagraph*{Number:} 1

\subparagraph*{Type:} Electrical real (TimeContinuousAmplitudeContinuousReal)

\subsection*{Output Signals}

\subparagraph*{Number:} 1

\subparagraph*{Type:} Electrical real (TimeDiscreteAmplitudeContinuousReal)

\subsection*{Examples} 

\subsection*{Sugestions for future improvement}


\end{document}