\documentclass{article}
\usepackage[utf8x]{inputenc}
\usepackage{amsmath}
\usepackage{float}
\usepackage{graphicx}
\usepackage{multicol}
\usepackage{multirow}
\usepackage[margin=1in]{geometry}
\usepackage{indentfirst}
\usepackage{amsfonts}
\graphicspath{{./images/}}

\begin{document}

\title{Progress report}
\date{August 2016}
\author{Daniel Pereira}
\maketitle

\section{Introduction}

This document describes the simulation of a Quantum Key Distribution (QKD) in back-to-back configuration. To that end, it includes a description of the system to be simulated and the individual code blocks.

\section{System to analyse}

In this step on our work we are attempting to simulate a \textit{back-to-back} QKD setup, in which the signal output by the emitter is directly fed into the receiver.
\par
For simplicity, in this early stage we have chosen to use a BPSK coding. This is a simplified setup but still allows us to study the effect of the relative phase between the local oscillator and the signal.
\par
A simplified diagram of the system being simulated is presented in the Figure~\ref{fig:physicalsystem}. In broad terms, our QKD simulation is required to be able to send a random bit sequence (the key) by means of an optical signal, accomplish a full key reconstruction from the optical signal and subsequently perform a key security check through the evaluation of the Bit Error Rate (BER).

\begin{figure}[h]
\centering
\includegraphics[width=.6\linewidth]{PhysicalSystem.png}
\caption{Overview of the QKD system being simulated.}
\label{fig:physicalsystem}
\end{figure}

In the next section we will show that our simulation can currently accomplished all of the proposed objectives.

\section{Numerical model}

The current simulation can be visually interpreted by the block diagram presented in Figure~\ref{fig:simulatedsystem}. In this section we will go into some detail on the workings of the blocks responsible for the Signal detection and Key reconstruction (i.e. all blocks other than \textbf{MQAM Transmitter} and \textbf{Local Oscillator}).

\begin{figure}[h]
\centering
\includegraphics[width=\linewidth]{simulatedsystem.png}
\caption{Visual representation of the simulated system.}
\label{fig:simulatedsystem}
\end{figure}

Along the description of each block, we include the crucial coding (the code responsible for setting the signal parameters is omitted) and a figure showing the output of each block.


\subsection{Balanced Beamsplitter}

This block takes two input signals, splits them in accordance with the beamsplitter matrix presented in~\eqref{eq:matrix}, and outputs two signals, effectively mixing the two inputs.

\begin{equation}\label{eq:matrix}
\frac{1}{\sqrt{2}}
\begin{pmatrix}
1 & 1\\
1 & -1
\end{pmatrix}
\end{equation}

The following code is run for every pair of values in the input signals.

\begin{verbatim}
		t_complex x1;
		inputSignals[0]->bufferGet(&x1);
		t_complex x2;
		inputSignals[1]->bufferGet(&x2);

		
		t_complex x4 = div*(x1 + x2);
		t_complex x3 = div*(x1 - x2);

		
		outputSignals[0]->bufferPut(x3);
		outputSignals[1]->bufferPut(x4);
\end{verbatim}

The inputs for this block are presented in the Figures~\ref{fig:bsin1}~and~\ref{fig:bsin2} and the outputs in the Figures~\ref{fig:bsout1}~and~\ref{fig:bsout2}.

\begin{figure}[H]
\centering
\includegraphics[width=\linewidth, trim= 0mm 100mm 0mm 100mm, clip]{sentsig.pdf}
\caption{Sent signal.}
\label{fig:bsin1}
\end{figure}

\begin{figure}[H]
\centering
\includegraphics[width=\linewidth, trim= 0mm 100mm 0mm 100mm, clip]{lo.pdf}
\caption{Sent local oscillator.}
\label{fig:bsin2}
\end{figure}

\begin{figure}[H]
\centering
\includegraphics[width=\linewidth, trim= 0mm 100mm 0mm 100mm, clip]{bs1.pdf}
\caption{First output of the beam-splitter.}
\label{fig:bsout1}
\end{figure}

\begin{figure}[H]
\centering
\includegraphics[width=\linewidth, trim= 0mm 100mm 0mm 100mm, clip]{bs2.pdf}
\caption{Second output of the beam-splitter.}
\label{fig:bsout2}
\end{figure}

\subsection{Photodiode}

This block takes one complex optical bandpass signal input and outputs one real electrical signal, taking into account the power of the input signal (correcting it in accordance with the Bandpass signal conversion) and the responsitivity of the photodiodes being simulated.
\par
The following code is run for every value in the input signal.

\begin{verbatim}
		t_complex x1;
		inputSignals[0]->bufferGet(&x1);
		t_real x1r = real(x1);
		t_real x1i = imag(x1);
		t_real radius = 0.0003;
		t_real E0 = 8.854187817e-12;

		t_real power = 2 * SPEED_OF_LIGHT * PI * radius*radius * E0 * (x1r*x1r + x1i*x1i) * 0.25;
		t_real current = 1 * power;

		outputSignals[0]->bufferPut(current);
\end{verbatim}

As shown in Figure~\ref{fig:simulatedsystem}, two photodiodes are used for the input of the Homodyne detector, taking for input the signals presented in Figures~\ref{fig:bsout1}~and~\ref{fig:bsout2}, these blocks output the signals presented in Figures~\ref{fig:photod1}~and~\ref{fig:photod2}.

\begin{figure}[H]
\centering
\includegraphics[width=\linewidth, trim= 0mm 100mm 0mm 100mm, clip]{photod1.pdf}
\caption{Output of the first photo-diode.}
\label{fig:photod1}
\end{figure}

\begin{figure}[H]
\centering
\includegraphics[width=\linewidth, trim= 0mm 100mm 0mm 100mm, clip]{photod2.pdf}
\caption{Output of the second photo-diode.}
\label{fig:photod2}
\end{figure}


\subsection{Subtractor}

This block takes two real electrical signals, subtracts them and outputs one real electrical signal.
\par
The following code is run for every pair of values in the input signals.


\begin{verbatim}
		t_real x1;
		inputSignals[0]->bufferGet(&x1);
		t_real x2;
		inputSignals[1]->bufferGet(&x2);

		t_real out = x1 - x2;

		outputSignals[0]->bufferPut(out);
\end{verbatim}

This block takes for input the signals presented in Figures~\ref{fig:photod1}~and~\ref{fig:photod2}, and outputs the signal presented in Figure~\ref{fig:subract}.

\begin{figure}[H]
\centering
\includegraphics[width=\linewidth, trim= 0mm 100mm 0mm 100mm, clip]{subtracted.pdf}
\caption{Subtraction of the two photo-currents.}
\label{fig:subract}
\end{figure}

\subsection{Amplifier}

This block takes one real electrical input and, amplifies it's amplitude and outputs one real electrical signal.
\par
The following code is run for every value in the input signal.

\begin{verbatim}
		t_real signalValue;
		inputSignals[0]->bufferGet(&signalValue);
		t_real out = signalValue*10000;

		outputSignals[0]->bufferPut(out);
\end{verbatim}

This block takes for input the signal presented in Figure~\ref{fig:subract}, and outputs the signal presented in Figure~\ref{fig:amplf}.

\begin{figure}[H]
\centering
\includegraphics[width=\linewidth, trim= 0mm 100mm 0mm 100mm, clip]{amplified.pdf}
\caption{Amplification of the photo-current.}
\label{fig:amplf}
\end{figure}

\subsection{Discretizer}

This block takes one real, continuous electrical input and samples it at a rate of one sample per bit period, outputting a real discrete signal.
\par
The following code is run for every value in the input signal.

\begin{verbatim}
		t_real signalValue;
		inputSignals[0]->bufferGet(&signalValue);
		auxint = auxint + 1;

		if (auxint == period)
		{
			auxint = 0;
			out = signalValue;
			outputSignals[0]->bufferPut(out);
		}
\end{verbatim}
 
Note that the discretization is accomplished by the if clause, that causes that only one value per bit period will be saved.
\par
This block takes for input the signal presented in Figure~\ref{fig:amplf}, and outputs the signal presented in Figure~\ref{fig:discrt}.

\begin{figure}[H]
\centering
\includegraphics[width=\linewidth, trim= 0mm 100mm 0mm 100mm, clip]{discretized.pdf}
\caption{Discretization of the amplified signal.}
\label{fig:discrt}
\end{figure}


\subsection{Delayer}

This block takes in one real discrete input and discards the first few values, such that the output signal is composed only by values related to the bit value being sent and not due to the convolution of sine waves.
\par
The following code is run for every value in the input signal.
\begin{verbatim}
		t_real signalValue;
		inputSignals[0]->bufferGet(&signalValue);
		auxint = auxint + 1;

		if (auxint >= delay)
		{
			out = signalValue;
			outputSignals[0]->bufferPut(out);
		}
\end{verbatim}

This block takes for input the signal presented in Figure~\ref{fig:discrt}, and outputs the signal presented in Figure~\ref{fig:delay}.

\begin{figure}[H]
\centering
\includegraphics[width=\linewidth, trim= 0mm 100mm 0mm 100mm, clip]{delayed.pdf}
\caption{Removal of the first samples.}
\label{fig:delay}
\end{figure}

\subsection{Bit decider}

This block takes in one real discrete input and outputs a binary string, attributing values in accordance to the relation between the input values and a internal reference value.
\par
The following code is run for every value in the input signal.
\begin{verbatim}
		t_real signalValue;
		inputSignals[0]->bufferGet(&signalValue);
		if (signalValue>0)
		{
			outputSignals[0]->bufferPut(1);
		}
		else
		{
			outputSignals[0]->bufferPut(0);
		}
\end{verbatim}

Note that in this case the reference value is merely 0, in the future this value will need to take into account the noise level.
\par
This block takes for input the signal presented in Figure~\ref{fig:delay}, and outputs the signal presented in Figure~\ref{fig:decision}.

\begin{figure}[H]
\centering
\includegraphics[width=\linewidth, trim= 0mm 100mm 0mm 100mm, clip]{recvkey.pdf}
\caption{Result of the bit decision, reconstructed key.}
\label{fig:decision}
\end{figure}



\subsection{Bit Error-Rate}

This block takes as inputs two binary strings and evaluates the number of coincidences, outputting a copy of of the inputs (this output is of no importance) and a .txt file with a report of the obtained results.
\par
The following code is responsible for the functioning of the block:
\begin{verbatim}
int ready = inputSignals[0]->ready();

	int space = outputSignals[0]->space();

	int process = min(ready, space);

	float NumberOfBits = recievedbits;
	float Coincidences = coincidences;

	float BER;
	BER = (NumberOfBits - Coincidences) / NumberOfBits * 100;

	if (process == 0)
	{
		ofstream myfile;
		myfile.open("BER.txt");
		myfile << "BER=" << BER<<"\%";
		myfile.close();
		return false;
	}



	for (int i = 0; i < process; i++) {

		t_binary signalValue;
		inputSignals[0]->bufferGet(&signalValue);
		t_binary SignalValue;
		inputSignals[1]->bufferGet(&SignalValue);

		recievedbits++;

		if (signalValue==SignalValue)
		{
			coincidences++;
		}



		outputSignals[0]->bufferPut(&signalValue);

	}
	return true;
\end{verbatim}
Note that the .txt file is only written when one the input signals is spent.
\par
This block takes for input the signals presented in Figures~\ref{fig:decision}~and~\ref{fig:sent}, and outputs the .txt file presented in Figure~\ref{fig:ber}.

\begin{figure}[H]
\centering
\includegraphics[width=\linewidth, trim= 0mm 100mm 0mm 100mm, clip]{sentkey.pdf}
\caption{Sent bit sequence.}
\label{fig:sent}
\end{figure}

\begin{figure}[H]
\centering
\includegraphics{ber.png}
\caption{Image of the report generated by the BER block.}
\label{fig:ber}
\end{figure}



%\bibliographystyle{unsrt}
%\bibliography{bibliography}

\end{document}