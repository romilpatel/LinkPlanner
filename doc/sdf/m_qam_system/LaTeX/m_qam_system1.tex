\documentclass[a4paper]{article}
\usepackage[top=1in, bottom=1.25in, left=1.25in, right=1.25in]{geometry}
\usepackage{amsmath}
\usepackage{multicol}
\usepackage{graphicx}
\usepackage[utf8]{inputenc}
\usepackage[english]{babel}
\setlength{\parskip}{0.03cm plus4mm minus3mm}
\RequirePackage{ltxcmds}[2010/12/07]

\usepackage{hyperref}
%opening
\title{MQAM system}

\begin{document}
	
	\maketitle
	
MQAM system is a complex block of code that simulates the modulation, transmission and demodulation of an optical signal using M-QAM modulation.
	
It is composed of four blocks: a transmitter, a receiver, a communication channel and a block that performs a Bit Error Rate (BER) measurement. The schematic representation of the system is presented in figure \ref{MQAM_system_block_diagram}.

\begin{figure}
	\centering
	\includegraphics[width=0.8\textwidth]{MQAM_system_block_diagram}
	\caption{Schematic representation of the MQAM system.}\label{MQAM_system_block_diagram}
\end{figure}

\subsection*{MQAM transmitter}

A complete description of the MQAM transmitter either block by block or as a whole can be found in the \textit{lib} repository. 

This block generates one or two optical signals. It also generates a binary signal that is used to perform a BER measurement.

\subsection*{MQAM receiver (homodyne receiver)}

A complete description of the MQAM transmitter either block by block or as a whole can be found in the \textit{lib} repository.

The MQAM receiver is a homodyne receiver. It accepts one input optical signal and outputs a binary signal. It performs the M-QAM demodulation of the input signal.

\subsection*{BER measurement}

\subsection*{Input parameters}

The input parameters of the system are the ones from the MQAM transmitter plus the ones from the MQAM receiver.

\end{document}