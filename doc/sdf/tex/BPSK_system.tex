\documentclass[a4paper]{article}

% subfile handling packages
\usepackage{subfiles}
\newcommand{\onlyinsubfile}[1]{#1}
\newcommand{\notinsubfile}[1]{}

% document packages
\usepackage[top=1in, bottom=1.25in, left=1.25in, right=1.25in]{geometry}
\usepackage{amsmath}
\usepackage{multicol}
\usepackage{graphicx}
\usepackage{multirow}
\usepackage{tabulary}
\usepackage{hhline}
\RequirePackage{ltxcmds}[2010/12/07]
\graphicspath{{../../images/}}
%\graphicspath
\usepackage{float}
\usepackage{amsfonts}
\usepackage{hyperref}
\usepackage{footnote}
\makesavenoteenv{tabular}

% Document metadata
\title{BPSK system}
\author{ }
\date{ }

\begin{document}
\renewcommand{\onlyinsubfile}[1]{}
\renewcommand{\notinsubfile}[1]{#1}

\maketitle

\section{Introduction}

This document describes the simulation BPSK system in back-to-back configuration. 

\section{Functional Description}

A simplified diagram of the system being simulated is presented in the Figure~\ref{fig:homodynesystem}. The system simulated takes a random binary string and encodes it in an optical bandpass signal, signal that afterwards is decoded in order to re-obtain the original binary string.
\par
The decoding of the optical signal is accomplished by an homodyne receiver, which combines the signal with a local oscillator with a user-determined phase. The homodyne receiver block output is then fed into a block that compares it with the original binary string and computes the Bit Error Rate (BER) along with it's upper and lower bounds for a certain user defined confidence level.

\begin{figure}[h]
\centering
\includegraphics[width=\linewidth]{bpskdiagram.png}
\caption{Overview of the BPSK system being simulated.}
\label{fig:homodynesystem}
\end{figure}

\begin{table}[H]
\centering
\begin{tabular}{c|c}
System Blocks    & netxpto Blocks   \\ \hline
BPSK Transmitter & MQamTransmitter  \\
BPSK Receiver    & HomodyneReceiver \\
BER Estimator    & BitErrorRate                      
\end{tabular}
\end{table}

\section{System Input Parameters}

This system takes into account the following input parameters:

\begin{savenotes}
\begin{table}[H]
\centering
\begin{tabulary}{1.0\textwidth}{|C|C|}
\hline
\textbf{System Parameters}      & \textbf{Description} 																 \\ \hline
NumberOfBits           & Gives the number of bits to be simulated		          										 \\ \hline  
BitPeriod              & Sets the time between adjacent bits                                                           \\ \hline 
SamplesPerSymbol       & Establishes the number of samples each bit in the string is given \footnotemark[1]	         \\ \hline
pLength                & PRBS pattern length					                      									 \\ \hline  
iqAmplitudesValues     & Sets the state constellation																	 \\ \hline  
outOpticalPower\_dBm   & Sets the optical power, in units of dBm, at the transmitter output							 \\ \hline  
LOoutOpticalPower\_dBm & Sets the optical power, in units of dBm, of the local oscillator used in the homodyne detector \\ \hline  
LocalOscillatorPhase   & Sets the initial phase of the local oscillator used in the homodyne detector					 \\ \hline  
TransferMatrix         & Sets the transfer matrix of the beam splitter used in the homodyne detector					 \\ \hline  
Responsivity           & Sets the responsivity of the photodiodes used in the homodyne detector						 \\ \hline  
Amplification          & Sets the amplification of the trans-impedance amplifier used in the homodyne detector			 \\ \hline  
NoiseAmplitude         & Sets the amplitude of the gaussian thermal noise added in the homodyne detector				 \\ \hline  
Delay                  & Sets the delay factor of the homodyne detector												 \\ \hline  
PosReferenceValue      & \multirow{2}{*}{Set the positive and negative reference values for the bit decision block}     \\ \cline{1-1}
NegReferenceValue      &                                                                                                \\ \hline
Confidence             & Sets the confidence interval for the calculated QBER                                           \\ \hline  
MidReportSize          & Sets the number of bits between generated QBER mid-reports                                     \\
\hline                   
\end{tabulary}
\end{table}		
\end{savenotes}	

\footnotetext[1]{Simulation time resolution = $\frac{\text{BitPeriod}}{\text{\text{SamplesPerSymbol}}}$}

\section{Inputs}

This system takes no inputs.

\section{Outputs}

This system outputs the following objects:
\begin{itemize}
\item Signals:
\begin{itemize}
\item Initial Binary String; (MQAM$_0$)
\item Optical Signal with coded Binary String; (S$_{00}$)
\item Decoded Binary String; (S$_{01}$)
\end{itemize}
\item Other:
\begin{itemize}
\item Bit Error Rate report in the form of a .txt file. (BER.txt)
\end{itemize}
\end{itemize}

\section{Simulation Results}

We consider the following scenarios:
\begin{itemize}
\item \ref{subsec:scenario1} Basic BPSK back to back with normally distributed thermal noise.
\end{itemize}

\subsection{BPSK with thermal noise}\label{subsec:scenario1}

The following results were obtained from the simulation using the following input parameters:
\begin{table}[H]
\centering
\begin{tabular}{rl}
NumberOfBits=           & 1000                                                     \\
SamplesPerSymbol=       & 16                                                       \\
pLength=                & 5                                                        \\
iqAmplitudesValues=     & \{ \{ 1, 0 \}, \{ -1, 0 \} \}                            \\
outOpticalPower\_dBm=   & -20                                                      \\
LOoutOpticalPower\_dBm= & -10                                                      \\
LocalOscillatorPhase=   & 0                                                        \\
TransferMatrix=         & \{ \{ 1/sqrt(2), 1/sqrt(2), 1/sqrt(2), -1/sqrt(2) \} \}  \\
Responsivity=           & 1                                                        \\
Amplification=          & 1e6                                                      \\
NoiseAmplitude=         & 15.397586549153788                                       \\
Delay=                  & 9                                                        \\
\end{tabular}
\end{table}

The system took the binary string presented in Figure~\ref{fig:sentkey} and encoded it into the optical signal in Figure~\ref{fig:sentsig}. Notice the BPSK constelation of the signal, presented in Figure~\ref{fig:constellation}.
\begin{figure}[H]
\centering
\includegraphics[width=\linewidth, trim= 0mm 95mm 0mm 95mm, clip]{binarystring.pdf}
\caption{Sent binary key.}
\label{fig:sentkey}
\end{figure}

\begin{figure}[H]
\centering
\includegraphics[width=\linewidth, trim= 0mm 95mm 0mm 95mm, clip]{sentsignal.pdf}
\caption{Sent signal.}
\label{fig:sentsig}
\end{figure}

\begin{figure}[H]
\centering
\includegraphics[width=\linewidth, trim= 0mm 95mm 0mm 95mm, clip]{constellation.pdf}
\caption{Constellation of the sent signal.}
\label{fig:constellation}
\end{figure}

Homodyne detection is then performed, using to that effect the local oscillator signal presented in Figure~\ref{fig:local}. Figures~\ref{fig:subtract}~and~\ref{fig:noisy} show the addition of noise to the signal.

\begin{figure}[H]
\centering
\includegraphics[width=\linewidth, trim= 0mm 95mm 0mm 95mm, clip]{localosc.pdf}
\caption{Homodyne receiver internal signal: local oscillator used for Homodyne detection.}
\label{fig:local}
\end{figure}

\begin{figure}[H]
\centering
\includegraphics[width=\linewidth, trim= 0mm 95mm 0mm 95mm, clip]{subtract.pdf}
\caption{Homodyne receiver internal signal: subtraction of the signals outputted by the photodiodes.}
\label{fig:subtract}
\end{figure}

\begin{figure}[H]
\centering
\includegraphics[width=\linewidth, trim= 0mm 95mm 0mm 95mm, clip]{noisy.pdf}
\caption{Homodyne receiver internal signal: amplification of the signal in Figure~\ref{fig:subtract} with added noise.}
\label{fig:noisy}
\end{figure}

The result of the homodyne detection is the binary string presented in~\ref{fig:decoded}, which is then compared to the original binary string by the BER block, which outputs the report presented in Figure~\ref{fig:ber}.

\begin{figure}[H]
\centering
\includegraphics[width=\linewidth, trim= 0mm 95mm 0mm 95mm, clip]{decodedbinarystring.pdf}
\caption{Decoded binary string, output of the Homodyne receiver block.}
\label{fig:decoded}
\end{figure}

\begin{figure}[H]
\centering
\includegraphics[width=\linewidth]{berreport.png}
\caption{Bit-Error-Rate report.}
\label{fig:ber}
\end{figure}

\pagebreak


\section{Block Description}

\subsection{MQAM Transmitter}
% subfile goes here

\subsection{Homodyne Receiver}
\subfile{../../lib/tex/homodyne_reciever}


\subsection{Bit Error Rate}
\subfile{../../lib/tex/ber}

\subsection{Local Oscillator}
\subfile{../../lib/tex/localoscillator}

\subsection{Beam Splitter}
\subfile{../../lib/tex/beamsplitter}

\subsection{Photodiode}
\subfile{../../lib/tex/photodiode}

\subsection{Subtractor}
\subfile{../../lib/tex/subtractor}

\subsection{Amplifier}
\subfile{../../lib/tex/amplifier}

\subsection{Discretizer}
\subfile{../../lib/tex/discretizer}

\subsection{Delayer}
\subfile{../../lib/tex/delayer}

\subsection{Bit Decider}
\subfile{../../lib/tex/decider}

\subsection{Bit Error Rate}
\subfile{../../lib/tex/ber}

\section{Known Problems}

\begin{itemize}
\item Homodyne Super-Block not functioning
\item MQAM Transmitter PDF needs to be written
\item 8 bits being lost of every signal
\item If the bit string length is larger than 512, this first 512 bits are lost
\end{itemize}

\end{document}